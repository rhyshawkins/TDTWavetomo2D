\documentclass[a4paper,12pt]{article}

\usepackage{natbib}
\usepackage{hyperref}

%\setlength{\textwidth}{425pt}
%\setlength{\oddsidemargin}{42pt}
%\setlength{\evensidemargin}{-15pt}

\setlength{\topmargin}{-15pt}

\begin{document}

\title{Wavetomo2D User Manual}
\author{Rhys Hawkins}
\date{April 2018}

\maketitle

\tableofcontents

\section{Introduction}

This software uses a Bayesian Trans-dimensional Tree\citep{Hawkins:2015:A} approach with a
wavelet parameterisation for the inversion of 2D surface wave
propagation in a rectangular region of the Earth. The target
application is Ambient noise tomography. It can perform both linear
great circle path inversions and fully non-linear inversions using an
inbuilt fast marching ray tracing code.

The software allows both hierarchical error estimation where there are
uncertainties in the error level on input observations and
hierarchical prior width. These two abilities combined allow the error
level and prior to be solved for as part of the inversion, resulting
in a more automated inversion.

Included is a parallel version that allows any combination of parallel
chains, parallel forward model evaluation and Parallel
Tempering\citep{Sambridge:2014:A}.

This software was developed by Rhys Hawkins as part of research leading
towards a PhD\citep{Hawkins:2018:A} at the Australian National University
and is Copyright 2014 - 2018 Rhys Hawkins and released under a GPL v3 Licence.

\section{Prerequisites}

It assumed that the software will be run on a Unix like system. This
software was developed and tested on Linux. The software requires a
recent GNU c++ compiler as some of the libraries use C++ templates
which are not compatible with older compilers.

The external libraries required by this software are

\begin{description}
\item[GSL] The GNU scientific library (2.4)
\item[GMP] The GNU arbitrary precision library (6.1)
\item[OpenMPI] Version 1.6 and 2.1 have been used.
\end{description}

\section{Installation}

The source code and supporting files are contained in the {\tt Wavetomo2dFrequencyInvert.tar.gz}
file, to extract and compile the code, use the following steps:

\begin{verbatim}
tar -xzf Wavetomo2dFrequencyInvert.tar.gz
cd Wavetomo2dFrequencyInvert/libraries
make
cd ..
make
\end{verbatim}

And the executables will be in the Wavetomo2DFrequencyInvert directory.

\section{Quick start tutorial : Single velocity map}

This section describes a simple walkthrough of running the code on
a synthetic data. The synthetic data is generated on a simple
model using a coarse checker board model with 16 stations regularly
spaced in a grid fashion.

The files for this inversion are located in the tutorial subdirectory.

\subsection{Preparing the input data file}

This part of tutorial is designed to demonstrate how to convert your
own data into the file format required for inversion. It is assumed
that the observations consist of a set of stations with interstation
travel times or observed velocities between selected station
pairs. Additionally, the great circle path distance is required between
stations as the path average velocity is used.

There is a python script available for this in the {\tt scripts}
subdirectory to convert a simple list of stations and list of interstation
path average velocities into the correct format for inversion. This
is the recommended method.

The list of stations file consists of a alphanumeric station code
followed by longitude and latitude coordinates on each line separated
by spaces, for example

\begin{verbatim}
XXX000    -7.500000000    -7.500000000
XXX001    -2.500000000    -7.500000000
XXX002     2.500000000    -7.500000000
\end{verbatim}

For the list of interstation observations, the file format is
station A code, station B code, then the interstation distance in
kilometres, the path average velocity in kilometres per second
and the uncertainty in the velocity. For example

\begin{verbatim}
XXX000 XXX001   551.215211031     3.111513582     0.100000000
XXX000 XXX002  1102.412475647     2.764755153     0.100000000
\end{verbatim}

The example two tutorial files are called {\tt tutorial\_stations.txt}
and {\tt tutorial\_observations.txt}.

These two files can be converted into an input file for inversion
using the script {\tt convertsingleobservations.py}, for example
running from the tutorial directory

\begin{verbatim}
python2 ../scripts/convertsingleobservations.py \
  -s tutorial_stations.txt \
  -v tutorial_observations.txt \
  -o tutorial_data.txt
\end{verbatim}

where the output is placed into {\tt tutorial\_data.txt}. The script is
a python version 2 script ({\tt python2} may need to be replaced with {\tt python} or {\tt python-2.7}
depending on your system.

Using this script is the recommended way of constructing an input file
for the inversion of a single velocity map. Alternatively you can
construct the input file using the description in Section \ref{sec:fileformat}.

\subsection{Creating a prior/proposal file}

The Trans-dimensional tree approach with a wavelet parameterisation adds and removes
wavelet coeffiecents to a model to represent an image. The wavelet coefficients are
by construction organised in a hierarchy from coarse to fine length scale. This
means the prior specification is difficult to tune. The recommended and simplest
approach to specification of the prior is to use either a fixed or hierarchical
Laplacian prior as this gives generally good performance and is far simpler to
specify.

The prior and proposals are specified in a single text file which we give
an example below (taken from {\tt tutorial/tutorial\_prior.txt})

\begin{verbatim}
laplace
0.1
priorbirth
depthgaussianperturb
7
0.020
0.020
0.025
0.035
0.050
0.050
0.050
\end{verbatim}

The first two lines specify a Laplacian prior with 0.2 width (similar to standard
deviation). This 0.2 value may need to be adjusted if a fixed Laplacian prior is
used in the inversion and birth/death acceptance is poor.

The next line specifies how birth/death proposals are performed, it is
suggested that this is always left as {\tt priorbirth} meaning new
wavelet coefficients are sampled from the prior.

The rest of the file specifies the proposal used for change of value
proposals for wavelet coefficients. The example here, {\tt
  depthgaussianperturb}, means that each scale length of wavelet
coefficients is perturbed with a Gaussian proposal with independently
specified standard deviation. In this example we have specified 7
levels, however if the wavelet model has more than 7 levels, it uses
the last level specified for higher levels so this specification
doesn't need to match the size of the wavelet model.

For tuning the acceptance rates of an inversion, this set of standard
deviations may need to adjusted to attempt to achieve an acceptance
rate between 20 and 50 percent. This is most important for the longer
scale length features and we will discuss the diagnostics in Section
\ref{sec:tutoriallinearized}.

This example file is a good starting point for ambient noise
tomography problems where the anomalies are of the order of 3.0km/s
$\pm$ 1.0km/s. For larger scale anomalies, the width of the Laplacian
prior might need to be tuned as well as the standard deviations of the
Gaussian perturbations.

\subsection{Linearized inversion}
\label{sec:tutoriallinearized}

\subsubsection{Introduction}

Once we have our input data file and a prior file, we can perform an
inversion. This software is structured to store the entire Markov
chain in a binary file from which results can be obtained using
post processing commands.

The command to do a single velocity map inversion is {\tt wavetomo2dfrequencysliceinvert}
(a single velocity map is known as a slice in this code). The
two main post processing commands for the single velocity inversions
are {\tt postprocess\_slice\_likelihood} and {\tt postprocess\_slice\_mean}.

The common pattern to running an inversion will be of the form

\begin{verbatim}
mkdir results_<run description>
wavetomo2dfrequencysliceinvert \
  -i <my data> \
  -M <my prior> \
  -o results_<run description>/  \
  ...
postprocess_slice_mean \
  -i results_<run description>/ch.dat \
  -o results_<run description>/mean.txt \
  ...
\end{verbatim}

where the ellipsis represent other options.

\subsubsection{Inversion}

Linearized inversions are very useful obtain a relatively quick inversion of a dataset.
On a reasonable computer, the inversion described here takes less than a minute to run.
From the {\tt tutorial} subdirectory, we can perform a linearized inversion with the
following set of commands where the options will be described in detail below

\begin{verbatim}
mkdir -p results_linearized
../wavetomo2dfrequencysliceinvert \
  -i tutorial_data.txt \
  -M tutorial_prior.txt \
  -o results_linearized/ \
  -s 0 \
  -x 5 -y 5 \
  -u 2 \
  -n -10.0 -N 10.0 \
  -a -10.0 -A 10.0 \
  -w 4 \
  -l 1.0 \
  -H 0.1 \
  -L 0.2 \
  -t 100000 \
  -k 100 \
  -E
\end{verbatim}

The first two input parameters are the input data file and prior/proposal file respectively.
The {\tt -o} option specifies an output prefix which must include a trailing ``/'' if this
is an output directory. This will write all output files to the {\tt results\_linearized}
directory.

The {\tt -s 0} specifies the slice in the input data file to use. It is not needed here but
if the input file contains data for more than one frequency, then this option specifies
the index of the frequency to perform the inversion on.

The next three parameters specify the grid resolution. As wavelet models need to be powers
of two, the width, {\tt -x} and height {\tt -y} of the model are specified in powers of
two. Hence ``{\tt -x 5 -y 5}'' means that the wavelet model is $2^5$ by $2^5$ or a 32 by 32
image. The width and height can be different although it is recommended that long thin
models where width and height are significantly different be avoided. The {\tt -u 2} option
specifies that the wavelet image is upscaled twice and the forward model computed on a
128 by 128 grid, i.e. 32 $\times$ 2 $\times$ 2 = 128.

The next two lines specify the longitude and latitude range of the domain. This must
encompass all the stations in the input file otherwise an error will occur. The
{\tt -n -10.0 -N 10.0} is the longitude range and {\tt -a -10.0 -A 10.0} is the latitude
range.

The {\tt -w 4} option specifies the wavelet basis to use. Allowed values are from 0
to 4 with 0 representing a blocky image to 4 representing a smooth image. A value of
4 is recommended for ambient noise tomography inversions.

The {\tt -l 1.0} and {\tt -H 0.1} control the hierarchical error estimation in the inversion.
In this software the error standard deviation used in the likelihood function is a
scaling of the data errors within the input data file, i.e.

\begin{equation}
  \sigma = \lambda \sigma_d
\end{equation}

where $\lambda$ is the hierarchical error scale term (inverted for
during inversion) and $\sigma_d$ is the data error in the input
file. The {\tt -l 1.0} specifies the initial scaling term, and the
{\tt -H 0.1} specifies the standard deviation of the Gaussian
perturbations used to invert for the scaling term. If the {\tt -H}
option isn't specified, then a fixed scaling term is used throughout
the inversion.

The {\tt -L 0.2} controls the hierarchical (sometimes hyper) prior estimation
on the inversion. This option specifies the standard deviation of prior
width change proposals. If this option is not specified, the prior is fixed
at the value specified in the prior/proposal file. The recommended
setting for inversion is a Laplacian prior with the hierarchical prior
enabled.

The {\tt -t 100000} specifies the inversion runs for 100000 iterations.
Burnin and thinning is controlled in the post processing steps than follow.

The {\tt -k 100} specifies the upper limit on the number of wavelet
coefficients, i.e. the prior on the number of wavelet coefficients in the
model is between 1 and 100 in this case.

Lastly the {\tt -E} switch specifies to use the linearized forward model.
By default the code uses a fast-marching forward model that while more accurately
represent physical propagation is far slower to run.

\subsubsection{Diagnostics}

During inversion, diagnostics are printed by default every 1000 iterations
to monitor the progress of the inversion, e.g.

\begin{verbatim}
  9000: 104.032548 25 dc 0.034582 lambda 1.321463:
BirthSlice:    453  59.161:  0.000   2.326  27.273  40.000  62.319  78.462 
DeathSlice:    457  53.392:  0.000   0.000  14.286  25.397  41.667  87.647 
ValueSlice:   8090  79.703: 45.707  56.197  77.404  81.414  88.985  92.077
HierarchicalSlice   1715/  3000  57.167
HierarchicalPriorSlice   2614/  3000  87.133
\end{verbatim}

The first line gives the iteration number (9000), the negative log likelihood
without the normalization or the misfit (104.03), the number of wavelet
coefficients (25) and the hierarchical scaling term lambda (1.32). The dc
component represents the current model difference from the mean
path average velocity of the observations.

The next three lines give the acceptance rates of the birth, death and
value proposals. The first integer value (i.e. 453,457,8090) is the
number of proposals.  The next column is the overall acceptance
rate. Following the ``:'' is the acceptance rate for each level or
increasing scale length from left to right of the wavelet
coefficients. Recall that the wavelet model was specified with {\tt -x
  5 -y 5} which means that we have levels from 0 .. 5 or 6 levels.
If the acceptance rates deviate significantly from around 20 to 50
percent, particularly for the first 3 levels, then the proposal
widths in the prior/proposal file should be tuned to achieve a
more optimal acceptance. As a general rule, increasing the
proposal width decreases the acceptance rate and vice-versa.

The last two lines give the acceptance rate for the hierarchical error
scaling parameter and the hierarchical prior width as accepted
count/proposed count and as a percentage.

\subsubsection{Residuals}

The output file {\tt residuals.txt} contains the mean residuals from the
ensemble for each path. An example of this file is shown below and
it consists of the two station codes, the reported interstation
distance, the mean residual for the velocity, the observed apparent velocity,
the mean residual normalized by the error level, the travel time residual
and lastly the observed travel time.

\begin{flushleft}
  \tiny
\begin{verbatim}
XXX000 XXX001   551.215211031     0.134494871     3.111513582     1.140492187    -7.340160687   177.153400268
XXX000 XXX002  1102.412475647    -0.069415889     2.764755153    -0.564549664    10.269112161   398.737831974
XXX000 XXX003  1653.573577505    -0.060978190     2.820260465    -0.494239491    12.957246043   586.319454542
XXX000 XXX004   555.974633223     0.073099750     3.087768688     0.616064773    -4.164085973   180.057086330
XXX000 XXX005   784.644432028    -0.017511978     3.013106539    -0.153958650     1.522336176   260.410450766
XXX000 XXX006  1239.082737857     0.086273806     3.134845550     0.709897628   -10.586594950   395.261175740
XXX000 XXX007  1751.579412159    -0.027640325     2.993994424    -0.236350209     5.451286405   585.030953337
\end{verbatim}
\end{flushleft}

This file can be used to check individual paths for outliers. For example, if a travel time residual
is large relative to the observed travel time, i.e. column 7 relative to column 8, then this observation
may be an outlier. The same applies for column 4 relative to column 5. The normalized residuals, when
plotted as a histogram should approximate a Gaussian distribution with a standard deviation of one.

A cautionary note here is that these residuals are average over the entire simulation
and if early parts of the Markov chain are unconverged, these residuals may be over-estimates.

\subsubsection{Post processing}

The inversion above outputs the full Markov chain of the inversion plus
some other diagnostic files. The chain is stored in the binary
{\tt ch.dat} file. To extract ensemble mean for example, we need
to run a post processing command such as 


\begin{verbatim}
../postprocess_slice_mean -i results_linearized/ch.dat \
  -Z results_linearized/zoffset.txt \
  -o results_linearized/mean.txt \
  -D results_linearized/stddev.txt \
  -s 50000 \
  -t 10 \
  -x 5 -y 5 -w 4
\end{verbatim}

The {\tt -i results\_linearized/ch.dat} specifies the chain history
to process. Wavelet models are stored as deviations from the mean
of the path average velocity observations and this offset is saved
to a z-offset file. We need to specify this when obtaining
mean models with the {\tt -Z results\_linearized/zoffset.txt} option.

The post processing can output different models and statistics from
the chain history, but in this example we request the mean
and standard deviation.

The {\tt -s 50000} option specifies that the first 50000 iterations are
skipped or thrown out as burnin samples. {\tt -t 10} says to only
use every 10th iteration to compute the mean model. Constructing mean
models needs to perform an inverse wavelet transform on each
model and can be quite slow for large dimension models. Using {\tt -t 10}
or larger can be used to give a quicker more approximate answer.

Lastly the dimension of the wavelet model and wavelet basis must be
set with {\tt -x 5 -y 5 -w 4} and must match the dimensions specified
in the inversion.

The output text file is a simple grid of numbers in text format that can
be plotted by many different plotting programs of your choice. A snippet
of code to quickly plot these using python is shown below.

\begin{verbatim}
import numpy
import matplotlib.pyplot as P

mean = numpy.loadtxt('results_linearized/mean.txt')
fig, ax = P.subplots()
ax.imshow(mean, vmin = 2.5, vmax = 3.5, cmap = 'RdBu')
P.show()
\end{verbatim}

For other diagnostic outputs that are used for convergence monitoring,
the tool {\tt postprocess\_slice\_likelihood} can be used. An example
usage is shown below

\begin{verbatim}
../postprocess_slice_likelihood -i results_linearized/ch.dat \
  -o results_linearized/likelihood.txt \
  -H results_linearized/hierarchical.txt \
  -L results_linearized/prior.txt \
  -K results_linearized/khistory.txt \
  -t 100
\end{verbatim}

The various output files above can be used to monitor convergence of
the Markov chain. When chains are converged, a qualitative guide is
that all of the above history plots appear ``flat'' or sample about a
single value. For more quantitative metrics, multiple chains can be
run and the ``r'' statistic computed between
chains\citep{Hawkins:2017:A}.

\subsection{Non-linear inversion}

\subsubsection{Introduction}

In the previous section the basics of the use of the software was
demonstrated in a linearized inversion. The benefits of a linearized
inversion is that the forward model is very quick to compute with the downside
being that it approximates the physics of surface wave propagation between
stations. Incorporated into this software is an integrated Fast Marching
option for updating ray paths at each iteration. This increases the
computational time dramatically but can result in improved recovery of
structure.

\subsubsection{Inversion}

A recommendation here is to use a first linearized inversion to converge
the solution quickly, then collect statistical information from
the the non-linear solution. To start from the previous linearized
solution, and run a non-linear inversion, the command is shown below.

\begin{verbatim}
mkdir -p results_nonlinear
../wavetomo2dfrequencysliceinvert \
  -i tutorial_data.txt \
  -I results_linearized/final_model.txt \
  -M tutorial_prior.txt \
  -o results_nonlinear/ \
  -s 0 \
  -x 5 -y 5 \
  -u 2 \
  -n -10.0 -N 10.0 \
  -a -10.0 -A 10.0 \
  -w 4 \
  -l 1.0 \
  -H 0.1 \
  -L 0.2 \
  -t 1000 \
  -v 10 \
  -k 100 \
\end{verbatim}

Most of the options remain the same. The {\tt -I
  results\_linearized/final\_model.txt} option is used to start the
inversion from the previous linearized inversion.  Missing is the
{\tt -E} switch which is used to enable the linearized forward model.
Lastly, due to the increased computational time, the number of
iterations is reduced to 1000 and the rate of status updates is
increased to every 10 iterations ({\tt -t 1000 -v 10}).

\subsubsection{Diagnostics and Post-processing}

Both the diagnostics and post processing are exactly the same for
the linear and non-linear inversion schemes.

\subsection{Parallel likelihood/chains/tempering inversion}

\subsubsection{Introduction}

For the tutorial dataset, it would be sufficient to invert this on a
desktop machine but for larger datasets, more thorough sampling may be
required or larger computation. The Parallel tempering version of the
code, {\tt wavetomo2dfrequencysliceinvert\_pt}, enables this on both
multi-processor desktops and int supercomputer environments.

This version uses MPI for parallel processing and the allocation
of processors is controlled by three parameters. We show a full
example command line below but here we show the three parameters
with other parameters removed for now.

\begin{verbatim}
mpirun -np 12 ../wavetomo2dfrequencysliceinvert_pt 
  ...
  -c 2 \
  -T 3 \
  ...
\end{verbatim}

The first parameter is the total number of processors to run
the application with and this is specified with the {\tt mpirun}
command as per usual.

The next parameter is the number of parallel chains run at the
same time. In the example above, we are running 2 parallel chains
at each temperature. Lastly, for Parallel Tempering, we
specify the number of temperature steps which in the example above
is 3. Parallel tempering can be disabled by set the number of
temperatures to 1, i.e. {\tt -T 1}.

The number of chains times the number of temperatures must be an
integer factor of the total number of processors. The formula
to remember is

\begin{equation}
  N_{\mathrm{processors}} = N_{\mathrm{chains}} \times
  N_{\mathrm{temperatures}} \times N_{\mathrm{processors\; per\; chain}}
\end{equation}

where $N_{\mathrm{processors}}$ is specified with the {\tt -np} option
for {\tt mpirun}, $N_{\mathrm{chains}}$ is specified with the {\tt -c}
option and $N_{\mathrm{temperatures}}$ with the {\tt -T}.  The number
of processors chain, $N_{\mathrm{processors\; per\; chain}}$ is then
defined implicitly and inbuilt into the software is the ability to
spread the forward model calculation across multiple processors to
increase the speed of the inversion. For example, running the previous
inversions again using the parallel version, we could use a command
line such as

\begin{verbatim}
mpirun -np 2 ../wavetomo2dfrequencysliceinvert_pt 
  ...
  -c 1 \
  -T 1 \
  ...
\end{verbatim}

to achieve an approximately two times speed up over the
non-parallel version, {\tt wavetomo2dfrequencysliceinvert}.

\subsubsection{Inversion}

Finally a full example of running an inversion with Parallel tempering and
parallel chains, analogous to the first linearized inversion is

\begin{verbatim}
mkdir -p results_parallel_linear
mpirun -np 12 ../wavetomo2dfrequencysliceinvert_pt \
  -i ../tutorial_data.txt \
  -M tutorial_prior.txt \
  -o results_parallel_linear/ \
  -s 0 \
  -x 5 -y 5 \
  -u 2 \
  -n -10.0 -N 10.0 \
  -a -10.0 -A 10.0 \
  -w 4 \
  -t 100000 \
  -v 10000 \
  -l 1.0 \
  -k 100 \
  -H 0.1 \
  -L 0.2 \
  -E \
  -c 2 \
  -T 3 \
  -m 5.0 \
  -e 10  
\end{verbatim}

If the desktop machine or machine you're testing on has fewer than 12 cores, then
the above will run slowly. In this case it may be worth either reducing the
number of steps, or reducing the number of processors/temperatures/chains.

Most of the details remain the same and the {\tt -c} and {\tt -T} parameters have
already been explained. The {\tt -m 5.0} sets the maximum temperature for the
tempered chains to 5 and temperatures will be logarithmically spaced between 1
and this maximum temperature. Larger temperatures will reduce the acceptance
rates of parallel tempering exchange swaps and vice-versa. The {\tt -e 10}
sets the rate of parallel tempering exchange attempts to one every 10 iterations.

\subsubsection{Diagnostics}

When running the above command, there is no output as parallel output is
written to log files. It is possible to monitor the progress of the inversion
with either

\begin{verbatim}
tail results_parallel_linear/log.txt-000
\end{verbatim}

or

\begin{verbatim}
tail -f results_parallel_linear/log.txt-000
\end{verbatim}

for continuous logging. For each process there is a log file indexed by processor
(i.e. MPI rank).


\subsubsection{Post processing}

In the {\tt results\_parallel\_lienar} subdirectory, rather than
a single {\tt ch.dat} file there will be a number of them with
three digit indices. These are numbered in order of ascending
temperature so that for the preceding example, {\tt ch.dat-000}
and {\tt ch.dat-001} will be for temperature 1 and
{\tt ch.dat-004} and {\tt ch.dat-005} will be for temperature
5 with the other two at an intermediate temperature.

It is possible to post process individual {\tt ch.dat-???} files
with the {\tt postprocess\_slice\_mean} program as before, but
for parallel chains there is an MPI version and it is run as
follows

\begin{verbatim}
mpirun -np 2 \
  ../postprocess_slice_mean_mpi \
  -i results_parallel_linear/ch.dat \
  -Z results_parallel_linear/zoffset.txt \
  -o results_parallel_linear/mean.txt \
  -D results_parallel_linear/stddev.txt \
  -s 50000 -t 5 \
  -x 5 -y 5 -w 4
\end{verbatim}

This is almost exactly the same as the previous single processor version
although we need to use {\tt mpirun} with the same number of processes
as the number of chains in our inversion. Rather than specifying the
{\tt ch.dat-???} files directly, the input ({\tt -i} parameter)
specifies the chain history prefix and the program attaches the
appropriate suffixes, e.g.  {\tt -000}, {\tt -001}, etc.

This program computes the means and standard deviations across chains
producing single mean and standard deviation files.

For the other diagnostics of the inversion obtained through post
processing, i.e.  the likelihood, hierarchical error, hierarchical
prior and k history, these need to be processed per chain history file
with the single processor program as averaging them makes little sense. For
example

\begin{verbatim}
../postprocess_slice_likelihood -i results_parallel_linear/ch.dat-000 \
  -o results_parallel_linear/likelihood.txt-000 \
  -H results_parallel_linear/hierarchical.txt-000 \
  -L results_parallel_linear/prior.txt-000 \
  -K results_parallel_linear/khistory.txt-000 \
  -t 100
\end{verbatim}

and the same can be repeated for other chain history files.

\subsubsection{Parallel Restarting}

If we wished to use the previous parallel linear inversion to seed a
parallel non-linear inversion, the command to use is similar to the
single processor case except the directory is specified as the
input rather than directly specifying the {\tt final\_model.txt}
file, i.e.

\begin{verbatim}
mkdir -p results_parallel_nonlinear
mpirun -np 12 ../../wavetomo2dfrequencysliceinvert_pt \
  -i tutorial_data.txt \
  -M tutorial_prior.txt \
  -I results_parallel_linear/ \
  -o results_parallel_nonlinear/ \
  -s 0 \
  -x 5 -y 5 \
  -u 2 \
  -n -10.0 -N 10.0 \
  -a -10.0 -A 10.0 \
  -w 4 \
  -t 10000 \
  -v 1000 \
  -l 1.0 \
  -k 100 \
  -H 0.1 \
  -L 0.2 \
  -c 2 -T 3 -m 5.0 -e 10
\end{verbatim}

where {\tt -I results\_parallel\_linear/} is the option specifying the
directory to load {\tt final\_model.txt-???} from.

\subsection{Summary}

This tutorial was designed to enable a practioner or user of this
software to quickly understand its basic usage. The general use of
this software follows a pattern of

\begin{enumerate}
\item Run test single processor inversions with the linearized forward
  model on a local desktop/laptop to tune the proposals etc.
\item Run a parallel linear inversion on a cluster computer to obtain
  an approximate inversion.
\item Run a final parallel non-linear inversion on a cluster computer
  starting from the previous approximate inversion to obtain a final inversion.
\end{enumerate}

Many cluster computing environments have queuing systems with job
time limits. In this case, some short initial runs of a small number of
iterations may need to be run to gauge an estimated number of iterations
that can be performed within time limits.

\section{Individual Program References}

\subsection{wavetomo2dfrequencysliceinvert}

\subsection{wavetomo2dfrequencysliceinvert\_pt}

\subsection{postprocess\_slice\_mean}

\subsection{postprocess\_slice\_likelihood}

\appendix

\section{Data file format}
\label{sec:fileformat}

The input file format is intended to encapsulate all observations in
an ambient noise tomography problem across a band of frequencies. The
format consists of four sections, a list of stations, a list of
frequencies, a list of ray paths to compute and finally the
interstation observations.  In the following subsections we describe
these four parts and develop a simple three station example. The
complete file is shown at the end.

\subsection{Station list}

The station list consists of a single line with an integer indicating
the number of stations, followed by each station listed on a separate
line with its short code, longitude and latitude. An example follows:

\begin{verbatim}
3
STA -5.0 -5.0
STB 0.0 5.0
STC 5.0 -5.0
\end{verbatim}

\section{Frequency list}

For single frequency inversions, this section is fairly arbitrary, but
the format is a single line with an integer indication the number of
frequencies, followed by the frequency in Hertz. The frequencies
should be in order. An example of this section, for a single frequency
inversion follows:

\begin{verbatim}
1
0.10
\end{verbatim}

This states that there is only a single frequency of 0.10 Hz (10s period).

\section{Ray paths}

This section describes the ray paths that need to be computed for the
inversion. Due to reciprocity, the fast marching method only needs to
compute a travel time field for some stations. For example, in the
three station example, computing the travel time field for station A
allows the evaluation of travel times between station A and stations B
and C.  Hence for the three stations, we only need to perform two
travel time field calculations.  The format for this is a single line
with an integer indicating the number of travel time field
calculations, then for each of these the line consists of the station
code where the travel time field calculation starts from, followed by
a number of receiving stations, then a list of the receiving station
codes. For the three station example, this would look like:

\begin{verbatim}
2
STA 2 STB STC
STB 1 STC
\end{verbatim}

This states that the forward model first needs to compute a travel
time field starting from STA and compute travel times to STB and STC,
and next compute a travel time field starting from STB and compute the
travel time to STC.

\section{Observations}

The last section are the observations. The format for this is a single line with an
integer indicating the number of interstation observations. Then for each observation,
the first line consists of the source station code, receiving station code, and interstation
distance in kilometres. This is followed by observations for each frequency of phase velocity
(or group). In the example, and a common usage pattern of this software where single frequencies
are inverted, there is only a singe line of observations. The observation line consists of three velocity
values of which only the first is currently used, followed by the error on the velocity. For example:

\begin{verbatim}
3
STA STB 100.0
3.0 3.0 3.0 0.1
STA STC 100.0
2.9 2.9 2.9 0.1
STB STC 100.0
2.8 2.8 2.8 0.1
\end{verbatim}

Here the velocity between STA and STB is 3.0 km/s with an error of 0.1 km/s.

\subsection{A complete example}

The complete file developed during the preceding sections is shown below:

\begin{verbatim}
3
STA -5.0 -5.0
STB 0.0 5.0
STC 5.0 -5.0
1
0.10
2
STA 2 STB STC
STB 1 STC
3
STA STB 100.0
3.0 3.0 3.0 0.1
STA STC 100.0
2.9 2.9 2.9 0.1
STB STC 100.0
2.8 2.8 2.8 0.1
\end{verbatim}

\bibliography{bibliography}
\bibliographystyle{plainnat}

\end{document}
